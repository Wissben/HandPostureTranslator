\section{Solution proposée}
\paragraph{}
Le problème ne pouvant être résolu à l'aide de techniques d'algorithmique classique(notamment à cause de sa trop forte complexité), le recours à l'apprentissage automatique s'est vu être la meilleure option.
\subsection{Apprentissage sur les données}
\paragraph{}
Après avoir analysée et traité les données, nous avons ensuite entamer la conception de notre modèle, étant donnée que nous avons utilisé deux approches pour le pré-traitement des donnés, la façon dont notre modèle va apprendre ces données pourrait différer, c'est pourquoi cette section sera diviser en deux sous-sections : \\

	\begin{tikzpicture}[node distance=2cm]
	\node (pro2a) [process, yshift=-0.5cm] {Pré-traitement des données};
	\node (pro2b) [process, below of=pro2a, xshift=-4cm] {Approche naïve};
	\node (pro2c) [process, below of=pro2a, xshift=5cm] {Approche par clustering};
	\node (pro2ca) [process, below of=pro2c,text centered, text width=3cm,minimum height=2cm] {Apprentissage \textbf{one-user-left}};
	\node (pro2ba) [process, below of=pro2b, xshift=-2cm,text centered, text width=3cm,minimum height=2cm] {Apprentissage par partitionnement aléatoire};
	\node (pro2bb) [process, below of=pro2b, xshift=2cm,text centered, text width=3cm,minimum height=2cm] {Apprentissage \textbf{one-user-left}};
	\draw [arrow] (pro2a) -- (pro2b);
	\draw [arrow] (pro2a) -- (pro2c);
	\draw [arrow] (pro2b) -- (pro2ba);
	\draw [arrow] (pro2b) -- (pro2bb);
	\draw [arrow] (pro2c) -- (pro2ca);
	\end{tikzpicture}
\paragraph{Apprentissage par partitionnement aléatoire}
Danse cette approche, nous avons décidé de partitionné les donnés dans leur intégralité en 3 sous ensembles : 
\begin{itemize}
	\item $A$ Ensemble des instances sur les quelles sera lancé l'apprentissage.
	\item $V$ Ensemble des instances pour contrôler l'avancement de l'apprentissage.
	\item $T$ Ensemble des instances qui servirons à l'évaluation de l'approximation fournie par le modèle après sont apprentissage 
\end{itemize}
