\section{Problématique}
\paragraph{}
Dans ce projet, nous avons choisi de tenter de réaliser une application qui permettra de traduire différents gestes de la main en un texte ou une action.\par
Cette traduction automatisée peut servir par exemple à 
\begin{itemize}[label=\textbullet]
	\item Orientation à distance d'un robot.
	\item Traduction du langage des signes pour faciliter la communication avec les muets.
	\item \textbf{Air-Gesture} Communiquer une action à sa maison, son téléphone ou sa voiture avec la main ...
\end{itemize}\par 
Il devient évident que réussir à traduire(avec un taux d'exactitude assez raisonnable) des gestes de la main en temps réel(ou différé) s'avère être une tache irréalisable avec des algorithmes classiques\footnote{Algorithmes naïfs n'ayant pas recours à l'intelligence artificielle pour la résolution de problèmes}, en raison de la complexité de la relation entre les données, cela nous a donc conduits à développer un module d'apprentissage automatique basé sur les réseaux de neurones \label{ProlemSolver}
 pour accélérer l'aide à la décision, plus de détails dans \ref{neuralNetSolition}