\section{Conclusion générale}
Au bout de mini-projet, nous avons appris à mettre en pratique les aspects théoriques vus en cours, à travers la réalisation d'une maquette d'application et l'utilisation d'un environnement d'apprentissage automatique avec réseau de neurones.
\par 
Cependant la plus grande difficulté se trouvait dans l'analyse des données et leurs pré-traitement, aidé par une descriptions fournie dans \cite{datasetDetails}, nous avons pu filtré ce données de deux manière différentes, les codifié et les essayer sur plusieurs architectures.\par
Après des tests concluants, et une intégration à l'application (certe simpliste)  réussie, nous avons donc atteint notre objectif premier qui était d'appliquer un modèle entrainé a réaliser une tache dans le but de résoudre un problème.\par 
Ce fut une expérience enrichissante pour nous, car nous avons du nous approfondir un peu plus  dans le domaine de l'apprentissage automatique, en espérant pouvoir améliorer nos connaissances et partager ce savoir.

\section{Répartition des taches}

\begin{table}[H]
	\centering
	\begin{tabular}{|l|c|c|}
		\hline
		\multicolumn{1}{|c|}{Tâche}                      & BOURAHLA Yasser & BENHADDAD Wissam \\ \hline
		Rédaction du rapport                             & 20\%            & 80\%             \\ \hline
		Écriture scripts de pré-traitement des donnés         & 50\%            & 50\%             \\ \hline
		Écriture scripts d'apprentissage    & 50\%            & 50\%             \\ \hline
		Réalisation de l'application                     & 90\%            & 10\%             \\ \hline
		Ré-arrangement des dossiers/fichier du projet final & 40\%            & 60\%             \\ \hline
	\end{tabular}
\end{table}