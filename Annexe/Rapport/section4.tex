\section{Solution proposée}
\paragraph{}
Le problème ne pouvant être résolu à l'aide de techniques d'algorithmique classique(notamment à cause de sa trop forte complexité), le recours à l'apprentissage automatique s'est vu être la meilleure option.
\subsection{Apprentissage sur les données}
\paragraph{}
Après avoir analysée et traité les données, nous avons ensuite entamer la conception de notre modèle, étant donnée que nous avons utilisé deux approches pour le pré-traitement des donnés, la façon dont notre modèle va apprendre ces données pourrait différer, c'est pourquoi cette section sera diviser en deux sous-sections : \\

	\begin{tikzpicture}[node distance=2cm]
	\node (pro2a) [process] {Pré-traitement des données};
	\node (pro2b) [process, below of=pro2a,minimum width = 5cm, xshift=-4cm] {Approche naïve};
	\node (pro2c) [process, below of=pro2a,minimum width = 5cm, xshift=5cm] {Approche par clustering};
	\node (pro2ca) [process, below of=pro2c,text centered, text width=3.5cm,minimum height=2cm] {Apprentissage avec partitionnement \textbf{one-user-left}};
	\node (pro2ba) [process, below of=pro2b, xshift=-2.5cm,text centered, text width=3.5cm,minimum height=2cm] {Apprentissage avec partitionnement aléatoire};
	\node (pro2bb) [process, below of=pro2b, xshift=2.5cm,text centered, text width=3.5cm,minimum height=2cm] {Apprentissage avec partitionnement \textbf{one-user-left}};
	\draw [arrow] (pro2a) -- (pro2b);
	\draw [arrow] (pro2a) -- (pro2c);
	\draw [arrow] (pro2b) -- (pro2ba);
	\draw [arrow] (pro2b) -- (pro2bb);
	\draw [arrow] (pro2c) -- (pro2ca);
	\end{tikzpicture}
	
\subsubsection{Codification}
Pour ce qui est de la codification, nous avons opté pour une la codification un-parmis-N(dans notre cas N = 5) pour toutes les approches:\\
\begin{table}[H]
	\centering
	\begin{tabular}{|c|l|c|c|c|c|c|}
		\cline{1-1} \cline{3-7}
		1 & $\rightarrow$ & 1 & 0 & 0 & 0 & 0 \\ \cline{1-1} \cline{3-7} 
		2 & $\rightarrow$ & 0 & 1 & 0 & 0 & 0 \\ \cline{1-1} \cline{3-7} 
		3 & $\rightarrow$ & 0 & 0 & 1 & 0 & 0 \\ \cline{1-1} \cline{3-7} 
		4 & $\rightarrow$ & 0 & 0 & 0 & 1 & 0 \\ \cline{1-1} \cline{3-7} 
		5 & $\rightarrow$ & 0 & 0 & 0 & 0 & 1 \\ \cline{1-1} \cline{3-7} 
	\end{tabular}
	\caption{Correspondance entre geste$_i$ et codif(geste$_i$)}
	
\end{table}


\subsubsection{Variation des architectures}
Pour trouver la meilleure architecture pour chacune des approches, nous avons écrit un script pour faire varier les différents paramètres, nous avons jugé bon de varier les suivants : 
\newpage
\begin{itemize}
	\item Nombre de couches cachées, en effet comme vu en cours et en TP, se contenter d'une seule couche cachées peut s'avérer être un handicap pour le bon apprentissage de neurones, mais en rajouter trop peut nous conduire à un sur-apprentissage\footnote{Phénomène ou le modèle s'adapte beaucoup trop bien aux données d'apprentissage au dépens des données de testes}, nous avons donc fixé ce paramètre au maximum de 3.
	
	\item Nombre de neurones par couches cachées, aussi étroitement lié à la complexité du problème, ce paramètre joue un rôle clé dans la puissance d'apprentissage du modèle, en contre partie, il est sujet au phénomène de sur-apprentissage(trop de neurones dans une couche caché), mais aussi à celui de sous-apprentissage\footnote{Phénomène ou le modèle n'arrive pas à apprendre la relation entres le données présentées durant l'apprentissage.}
	
	\item Fonction d'apprentissage(Optimiseur), Élément central lors de la phase d'apprentissage, se contenter d'une ou deux fonctions ne serait pas très judicieux d'un point de vue scientifique, en effet il existe une multitude d'optimiseurs adapté a différents types de problèmes(SGD\cite{Bottou2010},Adadelta\cite{adapt},RMSprop\cite{rms} ...)
\end{itemize}


\subsubsection{Apprentissage avec partitionnement aléatoire}\label{randomPartLearning}
Danse cette approche, nous avons décidé de partitionné les donnés dans leur intégralité en 3 sous ensembles : 
\begin{itemize}
	\item $A$ Ensemble des instances sur les quelles sera lancé l'apprentissage.
	\item $V$ Ensemble des instances pour contrôler l'avancement de l'apprentissage.
	\item $T$ Ensemble des instances qui servirons à l'évaluation de l'approximation fournie par le modèle après sont apprentissage 
\end{itemize}

\begin{table}[H]
	\centering
	\begin{tabular}{|
			>{\columncolor[HTML]{67FD9A}}c |
			>{\columncolor[HTML]{67FD9A}}c |
			>{\columncolor[HTML]{67FD9A}}l |
			>{\columncolor[HTML]{67FD9A}}l |}
		\hline
		\multicolumn{4}{|c|}{\cellcolor[HTML]{67FD9A}}                                                                                                                                                        \\
		\multicolumn{4}{|c|}{\cellcolor[HTML]{67FD9A}}                                                                                                                                                        \\
		\multicolumn{4}{|c|}{\cellcolor[HTML]{67FD9A}}                                                                                                                                                        \\
		\multicolumn{4}{|c|}{\cellcolor[HTML]{67FD9A}}                                                                                                                                                        \\
		\multicolumn{4}{|c|}{\cellcolor[HTML]{67FD9A}}                                                                                                                                                        \\
		\multicolumn{4}{|c|}{\cellcolor[HTML]{67FD9A}}                                                                                                                                                        \\
		\multicolumn{4}{|c|}{\multirow{-7}{*}{\cellcolor[HTML]{67FD9A}Données d'apprentissage 75\%}}                                                                                                               \\ \cline{2-3}
		\multicolumn{1}{|l|}{\cellcolor[HTML]{67FD9A}}                   & \multicolumn{2}{c|}{\cellcolor[HTML]{FD6864}}                                         & \cellcolor[HTML]{67FD9A}                   \\
		\multicolumn{1}{|l|}{\multirow{-2}{*}{\cellcolor[HTML]{67FD9A}}} & \multicolumn{2}{c|}{\multirow{-2}{*}{\cellcolor[HTML]{FD6864}Données de validations 15\% }} & \multirow{-2}{*}{\cellcolor[HTML]{67FD9A}} \\ \cline{2-3}
		\multicolumn{4}{|l|}{\cellcolor[HTML]{67FD9A}}                                                                                                                                                        \\ \hline
		\multicolumn{4}{|c|}{\cellcolor[HTML]{34CDF9}}                                                                                                                                                        \\
		\multicolumn{4}{|c|}{\multirow{-1}{*}{\cellcolor[HTML]{34CDF9}Données de testes 30\%}}\\                                                                                                                     
		\multicolumn{4}{|c|}{\cellcolor[HTML]{34CDF9}}
		\\ \hline
	\end{tabular}
	\caption{Partitionnement des données}
		
\end{table}
\newpage
\subsubsection{Apprentissage avec partitionnement one-left-user}\label{oneLeftLearning}
\paragraph{}L'idée (comme suggérée dans \cite{datasetDetails} )est de séparer l'ensemble des instances $D$ en deux sous ensembles \textbf{Train} et \textbf{Test} en fonction de l'identifiant d'une utilisateur $id$ tel que : 
\begin{itemize}
	\item \textbf{Train} contient l'ensemble des instances ou $D[user]\ne id$
	\item \textbf{Test} contient l'ensemble des instances ou $D[user] = id$
\end{itemize}
\par 
Un utilisateur servira donc de testeur pour le modèle après apprentissage.
\begin{table}[H]
	\centering
	\begin{tabular}{|c|c|c|lllll}
		\cline{1-3} \cline{6-8}
		\textbf{ID Utilisateur}   & \textbf{...}                & \textbf{..}                 &                    & \multicolumn{1}{c|}{} & \multicolumn{1}{c|}{\cellcolor[HTML]{9AFF99}8} & \multicolumn{1}{c|}{\cellcolor[HTML]{9AFF99}...} & \multicolumn{1}{c|}{...} \\ \cline{1-3} \cline{6-8} 
		\cellcolor[HTML]{9AFF99}8 & \cellcolor[HTML]{9AFF99}... & \cellcolor[HTML]{9AFF99}... &                    & \multicolumn{1}{c|}{} & \multicolumn{1}{c|}{\cellcolor[HTML]{9AFF99}8} & \multicolumn{1}{c|}{\cellcolor[HTML]{9AFF99}...} & \multicolumn{1}{c|}{...} \\ \cline{1-3} \cline{6-8} 
		...                       & ...                         & ...                         &                    & \multicolumn{1}{c|}{} & \multicolumn{1}{c|}{\cellcolor[HTML]{9AFF99}8} & \multicolumn{1}{c|}{\cellcolor[HTML]{9AFF99}...} & \multicolumn{1}{c|}{...} \\ \cline{1-3} \cline{6-8} 
		...                       & ...                         & ...                         & \multirow{-4}{*}{} & \multicolumn{1}{c|}{} & \multicolumn{1}{c|}{\cellcolor[HTML]{9AFF99}8} & \multicolumn{1}{c|}{\cellcolor[HTML]{9AFF99}...} & \multicolumn{1}{c|}{...} \\ \cline{1-3} \cline{6-8} 
		\cellcolor[HTML]{9AFF99}8 & \cellcolor[HTML]{9AFF99}... & \cellcolor[HTML]{9AFF99}... & \multicolumn{4}{l}{$\rightarrow$}                                                                                                              &                          \\ \cline{1-3} \cline{6-8} 
		...                       & ...                         & ...                         &                    & \multicolumn{1}{l|}{} & \multicolumn{1}{l|}{...}                       & \multicolumn{1}{l|}{...}                         & \multicolumn{1}{l|}{...} \\ \cline{1-3} \cline{6-8} 
		\cellcolor[HTML]{9AFF99}8 & \cellcolor[HTML]{9AFF99}... & \cellcolor[HTML]{9AFF99}... &                    & \multicolumn{1}{l|}{} & \multicolumn{1}{l|}{...}                       & \multicolumn{1}{l|}{...}                         & \multicolumn{1}{l|}{...} \\ \cline{1-3} \cline{6-8} 
		\cellcolor[HTML]{9AFF99}8 & \cellcolor[HTML]{9AFF99}... & \cellcolor[HTML]{9AFF99}... &                    & \multicolumn{1}{l|}{} & \multicolumn{1}{l|}{...}                       & \multicolumn{1}{l|}{...}                         & \multicolumn{1}{l|}{...} \\ \cline{1-3} \cline{6-8} 
		...                       & ...                         & ...                         &                    & \multicolumn{1}{l|}{} & \multicolumn{1}{l|}{...}                       & \multicolumn{1}{l|}{...}                         & \multicolumn{1}{l|}{...} \\ \cline{1-3} \cline{6-8} 
		...                       & ...                         & ...                         & \multirow{-5}{*}{} & \multicolumn{1}{l|}{} & \multicolumn{1}{l|}{...}                       & \multicolumn{1}{l|}{...}                         & \multicolumn{1}{l|}{...} \\ \cline{1-3} \cline{6-8} 
	\end{tabular}
\end{table}
